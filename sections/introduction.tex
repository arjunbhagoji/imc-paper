\section{Introduction}\label{sec:related}

- Key questions about sharing of benign data alone:
- If homogeneous, more data will not hinder performance? \\
- If heterogeneous, expressive enough (deep) models should be able to learn, what happens with shallow models?\\
- Questions about malicious data sharing:
- Not shared for unsupervised
- Shared along with labels for S and SS

Contributions:
1. Clearly map out a taxonomy of data sharing for network security
2. Establish baselines when data sharing is helpful (with regards to data, model and sharing)
3. Practical method to model and understand data heterogenity


- What's missing? Different modes of data sharing, different learning paradigms, unclear what the performance gains from sharing are \\
- This paper characterizes in what scenarios data sharing can help:
- Data heterogenity when building a model of `normal behavior' (affects FP rates, will not affect TP) \\
- Modeling data het. is an open problem, especially for different classes of benign data
- Different learning paradigms: how is unsupervised learning aided by more data?\\
- Robustness to errors in labeling (ending up with malicious data in benign split)

- Experiment structure:
- Datasets: IDS, IoT, BAS
- Models: RF, IF
- Direct data sharing: S, US, SS; tune data het. for each
- Model sharing: S, US, SS
- Interpretation of results for different attack types