\section{Introduction}\label{sec: intro}
- Paragraph 1: Why is collaborative anomaly detection important?

- Paragraph 2: State of current work and need for this paper: need to understand the performance gains from different sharing modes (data, model, output) across learning paradigms. 

- Paragraph 3: Data volume, heterogeneity and noise. Discuss division of data sharing into benign and malicious \\ 
- Quantifying data  heterogeneity \\
- Benign data: If homogeneous, more data will not hinder performance? If heterogeneous, expressive enough (deep) models should be able to learn, what happens with shallow models?\\
- Malicious data: How is it shared?


- Paragraph 4: What are the possible learning paradigms and how do they interact with the data characteristics? \\
- Unsupervised learning: how is a model of `normal' behavior affected by multiple data sources? (affects FP rates, will not affect TP?)

- Paragraph 5: Experiment structure:
- Datasets: IDS, IoT, BAS \\
- Models: RF, IF\\
- Direct data sharing: S, US, SS; tune data het. for each\\
- Model sharing: S, US, SS\\
- Interpretation of results for different attack types\\


Contributions:
1. Clearly map out a taxonomy of data sharing for network security
2. Establish baselines when data sharing is helpful (with regards to data, model and output)
3. Practical method to model and understand data heterogenity


